\documentclass{report}

\usepackage{color}
\usepackage[margin=1in]{geometry}

%opening
\title{Markenrecht}
\author{Diana Burkart}

\begin{document}

\maketitle
\newpage

\tableofcontents
\newpage

\chapter{Vorlesungsinformationen}
21.04.2022
\begin{itemize}
	\item Informationen im Lehrbuch: Berlit, Wolfgang: Markenrecht, Verlag C.H.Beck, ISBN 3-406-53782-0, neueste Auflage.
	\item Bearbeitung Gesetzestext
	\begin{itemize}
		\item Keine Wörter hineinschreiben/ ergänzen (auch nicht 1., 2., 3.)
		\item Markierungen und Paragrafen hineinschreiben erlaubt
	\end{itemize}
	\item Schriftliche Klausur
	\item Problematisch in Klausur:
	\newline Sachen, die gefühlt verstanden sind, auszudrücken $\rightarrow$ Verstandenes versuchen selbst auszudrücken
\end{itemize}
28.04.2022
\begin{itemize}
	\item Klausurvorbereitung: Überlegen, wie kann man nach den Themen fragen.
	\newline Beispielfragen aufschreiben und selbst schriftlich beantworten.
\end{itemize}

\chapter{Mögliche Prüfungsfragen}
21.04.2022
\begin{itemize}
	\item Inwiefern ist Markenrecht absolutes Recht?
	\item Markenfunktionen ausdrücken können, nicht nur auflisten
\end{itemize}
28.04.2022
\begin{itemize}
	\item Von welchem Rechtsgebiet ist Markenrecht ein Teilgebiet?
	\begin{itemize}
		\item Gewerblicher Rechtsschutz
	\end{itemize}
	\item Was ist ein absolutes Recht?
	\begin{itemize}
		\item vs. relatives Recht: Forderungen (nur zwischen bestimmten Parteien)
		\item Gilt gegenüber jedermann
		\item z.B. Eigentum
		\item Benutzungsbefugnisse und Ausschließungsbefugnisse
	\end{itemize}
	\item Ist gewerblicher Rechtsschutz Teil des Privatrecht oder öffentliches Recht?
	\begin{itemize}
		\item Subjektives Privatrecht (Markenrecht)
		\item Könnte auch zum öffentlichen Recht gehören (Verwaltungsakt)
	\end{itemize}
	\item Wie entsteht Markenschutz?
	\begin{itemize}
		\item MarkenG §4
	\end{itemize}
	\item Was versteht man unter dem Grundsatz der Priorität?
	\begin{itemize}
		\item FCFS
	\end{itemize}
	\item Was versteht man unter dem Produktbezug der Marke?
	\begin{itemize}
		\item siehe 21.04.2022
		\item Angemeldegebühr für ersten 3 Klassen, dann für jede weitere Klasse (100€)
	\end{itemize}
\end{itemize}
\newpage

\chapter{Einführung: Theorie und System des Gewerblichen Rechtsschutzes}
21.04.2022
\begin{itemize}
	\item Gesetz gegen unlauteren Wettbewerb: Wettbewerbsrecht (gesetzliche Regel)
	\newline Lauterkeitsrecht + Kartellrecht = Wettbewerbsrecht
	\newline $\rightarrow$ Wettbewerb sichern
	\item Lauterkeitsrecht: allgemeine Spielregeln für Wettbewerb
	\newline Fair verhalten, nicht eigene Waren für die eines anderen ausgeben
	\item Kennzeichenrecht: Gewerbsleistung (besondere Ausprägung des Wettbewerbsrechts)
	\item Urheberecht (kein gewerbliches Schutzrecht): künstlerischer und nicht gewerblicher Bereich
	\item Gewerbewesen fördern, Entfaltung der Persönlichkeit auf gewerblicher Ebene fördern
	\item Intressen in den Vordergrund stellen (wirtschaftlich)
	\item Funktionierender Wettbewerb (im Interesse der Allgemeinheit \& des Einzelnen)
	\item Ausbeutung und Nachahmung gesetzlich nicht geschützter Erzeugnisse nicht geschützt
	\item Generalklausel: Sittenwidriges Handlen im Wettbewerb immer verboten
	\item Rechte werden immer dem zugeordnet, der sie zuerst anmeldet (Prioritätsprinzip: FCFS)
	\item Einhaltung von Formalitäten (Antrag)
\end{itemize}

\section{Gemeinsame Grundsätze}
21.04.2022
\begin{itemize}
	\item Immaterielle Güter:
	\newline unkörperlicher Gegenstand, nicht greifbares, geistiges Gut
	\begin{itemize}
		\item vs. Materielle Güter: Sachen, körperliche Gegenstände (Gegenteil zu immateriellen Gütern)
		\item Schutz von geistigem Gut, im Rechtsverkehr verselbstständigt
		\item Schutzwirkung überträgt sich auch auf körperliche Sachen, welche geistiges Eigentum enthält
		\item Geistiges Eigentum: Eigentum an immateriellem Gut
	\end{itemize}
	\item Subjektives Privatrecht
	\begin{itemize}
		\item Interessen selbst wahrnehmen und gegen andere durchsetzen
		\item vs. Objektives Recht: Gesamtheit aller Rechtsvorschriften (geschrieben + Gewohnheitsrecht)
		\item Einer Person zugeordnet
	\end{itemize}
	\item Dingliches Recht (vs. Schuldrecht)
	\begin{itemize}
		\item z.B. Eigentum
		\item Recht an einem Gegenstand
		\item vs. Schuldrecht: Forderung nur an bestimmten Vertragspartner
		\item Sachrecht
	\end{itemize}
	\item Absolutes Recht
	\begin{itemize}
		\item vs. Relatives Recht
		\item Benutzungsbefugnisse: Schutzrechtsinhaber an Schutzgegenstand
		\newline Befugnisse grundsätzlich zugewiesen (im Einzelfall nicht anwendbar)
		\item Negativ: Befugnis unerlaubte Benutzung abzuwehren (Ausschluss, ...)
	\end{itemize}
\end{itemize}

\section{Wesen der gewerblichen Schutzrechte}
21.04.2022
\begin{itemize}
	\item Zustandekommen und Abwehr von Verletzungen geistigen Eigentums
	\item Gebiete:
	\begin{itemize}
		\item Patentrecht, Gebrauchsmuster, Geschmacksmuster
	\end{itemize}
\end{itemize}

\section{Stellung des gewerblichen Rechtsschutzes im Rechtssystem}
21.04.2022
\begin{itemize}
	\item \textbf{Verhältnis zur Verfassung} (Grundgesetz)
	\begin{itemize}
		\item Artikel vs. Paragraphen
		\item Artikel 1 und 2 (wichtigste)
		\item Patentrecht:
		\newline Erfindung zu Kriegszwecken verwenden nicht möglich, wenn Erfinder das nicht möchte
		\item Wettbewerbsfreiheit (Artikel 3)
		\item Artikel 12: Berufsfreiheit (Spezielle Ausprägung von Artikel 2)
		\item Artikel 14: Schutz des Eigentums (Markenrecht, Ausschließlichkeitsrecht, Patentrecht $\rightarrow$ Eigentum)
		\newline $\rightarrow$ Freiheit im vermögensrechtlichen Raum
		\newline \textcolor{cyan}{Unabhängigkeit durch Vermögen und freie Möglichkeiten}
		\item Markenrecht monopolisiert Marke zu Gunsten des Markeninhabers (gesetzliches, rechtliches Monopol)
	\end{itemize}
	\item \textbf{Verhältnis zum Verwaltungsrecht}
	\item \textbf{Verhältnis zum bürgerlichen Recht}
	\item Rechtsordnung: Verschiedene Rechtsgebiete
	\begin{itemize}
		\item Privatrecht
		\begin{itemize}
			\item Besteht ausBGB, Handelsrecht \& kleinere (z.B. Arbeitsrecht)
			\item Regelt Rechtsbeziehungen von Bürgern untereinander (z.B. Vertrag) $\rightarrow$ Aushandeln
		\end{itemize}
		\item Öffentliches Recht
		\begin{itemize}
			\item Besteht aus Verfassungs- \& Verwaltungsrecht
			\item Verwaltungsrecht: Staat bestimmt etwas (kein Aushandeln) (z.B. Baugenehmigung, Exmatrikulation)
			\newline $\rightarrow$ von oben nach unten $\rightarrow$ Verwaltungsakt (kein Vertrag)
			\newline Wehren gegen Verwaltungsakt möglich
			\item Markenrecht (Verwaltungsbehörde $\rightarrow$ Antragsstellung)
			\newline DPMA erlässt Verwaltungsakte (ordentliche Gerichtbarkeit vs. Verwaltungsgerichtbarkeit)
		\end{itemize}
		\item Strafrecht
		\begin{itemize}
			\item Im öffentlichen Recht keine Verhandlungen, hier: Aushandeln möglich
		\end{itemize}
	\end{itemize}
\end{itemize}

\chapter{Das Markenrecht}
21.04.2022
\begin{itemize}
	\item Bezeichnungen, welche zur Kennzeichnung von Waren verwendet werden, vor Benutzung von Dritten schützen $\rightarrow$ Verwechslungen
	\item Wer Waren herstellt oder vertreibt kann diese mit willkürlich gewähltem Zeichen versehen, um diese von Waren anderer Hersteller zu unterscheiden
	\item Informationsgesellschaft $\rightarrow$ Information wird immer wichtiger
	\newline $\rightarrow$ Marke (im Verhältnis zum Produkt) wird immer wichtiger
	\item Ohne Marke hat Verbraucher kaum die Chance von Produkt Kenntnis zu erlangen
	\item Kennzeichenschutz: Grundlegende Anliegen des Wirschaftsrechts
	\newline Abnehmer kann nur frei entscheiden, wenn er auch unterscheiden kann
	\newline Entscheidung für Produkt setzt Möglichkeit/ Fähigkeit zu Unterscheiung voraus
	\item Unterscheidungsfunktion (Hauptfunktion)
	\item Anbieter: Qualitätsstandards und Werbebotschaften über Informationskanal (Marke), Informationsträger für Produkt
	\item Markenerfolg ist messbar
	\item Immaterieller Wert des Unternehmens (Marke)
	\newline Markenwert bestimmt häufig mehr als Hälfte des Unternehmenswertes
	\item Gründe für steigende Bedeutung der Marke und Markenschutzes: Globalisierung
	\newline Internationaler Erfolg eines Unternehmens nur mit international starker Marke
	\item Marke für internationalen und nationalen Wettbewerb unverzichtbar
	\item Schadensersatzansprüche und Unterlassungsanspruch aus BGB
	\newline (Verletzung eines Sonstigen Recht $\rightarrow$ gewerbliche Schutzrechte)
	\newline Kennzeichenrecht (§12 BGB): Beseitigung des Schadens
	\item \textit{Frage: was muss man tun, um eine Marke zu schützen, um gegen Angriffe vorzugehen?}
\end{itemize}

\section{Gesetzliche und historische Grundlagen}
21.04.2022
\begin{itemize}
	\item Vorgänger: Warenzeichengesetz von 1894
	\item Aktuelles Markengesetz grundlegend seit 1995
	\newline davor fast unverändert bis 1995
	\item EU Richtlinien: Vereinheitlichen Markenrecht in EU
	\begin{itemize}
		\item Gemeinschaftsmarkenverordnung
		\newline $\rightarrow$ EU Marken, einheitlich, für gesamte EU verbindlich
		\item EU Markenverordnung Modernisierung (Unions Marken)
	\end{itemize}
\end{itemize}

\subsection{Grundstrukturen und Grundbegriffe des Markenrechts}
21.04.2022

\subsection{Einordnung des Markenrechts}
21.04.2022
\begin{itemize}
	\item Einordnung in Rechtssystem:
	\newline Markenrecht ergänzt Firmenschutz (HGB + Markengesetz)
	\item Teilgebiet des gewerblichen Rechtsschutzes
\end{itemize}

\subsection{Gegenstand des Markenschutzes}
21.04.2022
\begin{itemize}
	\item Bezeichnungen, welche zur Kennzeichnung von Waren verwendet werden, vor Benutzung von Dritten schützen $\rightarrow$ Verwechslungen
	\item Wer Waren herstellt oder vertreibt kann diese mit willkürlich gewähltem Zeichen versehen, um diese von Waren anderer Hersteller zu unterscheiden
\end{itemize}

\subsection{Grundsatz der Priorität}
21.04.2022
\begin{itemize}
	\item Priorität bei Interessenkonflikt zwischen Inhabern
	\item Grundsatz des Vorrangs $\rightarrow$ Prioritätsjüngeres weicht prioritätsälterem Recht \textcolor{blue}{(FCFS)}
\end{itemize}

\subsection{Produktbezug der Marke}
21.04.2022
\begin{itemize}
	\item Kein abstrakter Schutz
	\item Beschränkung auf bestimmte Produkte (Produktklassen)
\end{itemize}

\subsection{Funktionen von Marken}
21.04.2022
\begin{itemize}
	\item Marke als Marketingmittel
	\item Herkunftsfunktion: Marke kennzeichnet Hersteller
	\newline $\rightarrow$ Marken, die Hersteller nicht kennzeichnen, werden nich geschützt
	\item Unterscheidungsfunktion: Unterscheidung von Herstellern
	\item Verbraucherfunktion: Produkte haben ähnliche Qualität (Vertrauensfunktion)
	\item Kommunikationsfunktion/ Werbefunktion: Imageträger der Hersteller
\end{itemize}

\subsection{Internationaler Schutz der Marken}
21.04.2022
\begin{itemize}
	\item Nur nationaler Schutz
	\item Internationaler Schutz muss überall sonst angemeldet werden
	\item Internationaler Schutz: EU Marke, Internationale Marke
\end{itemize}

\section{Inhalte und Umfang des Markenschutzrechts}
28.04.2022
\begin{itemize}
	\item Ablauf einer Markeneintragung
	\begin{itemize}
		\item DPMA Eintragung in Formblatt
		\item Prüfung durch DPMA
		\item Entscheidung positiv: Eintragung
		\newline sont Zurückweisungsbescheid
	\end{itemize}
	\item Markenregister
	\begin{itemize}
		\item Einsicht von jedem jederzeit möglich
		\item Bereits verwendete Marken können nicht mehr verwendet werden
	\end{itemize}
	\item Unbestimmte Rechtsbegriffe
	\begin{itemize}
		\item Unterscheidungskraft (Orginalität)
		\begin{itemize}
			\item Nicht orginell, wenn Produkt beschrieben wird
			\item Entscheidung "nach pflichtgemäßem Ermessen"
		\end{itemize}
	\end{itemize}
	\item Markenfähig: gemäß §3
	\begin{itemize}
		\item \textcolor{green}{Markieren}: "Als Marken können alle Zeichen [...] geschützt werden"
		\item Abstrakte Unterscheidungskraft, abstrakte Markenfähigkeit
		\item Gilt auch für Benutzungsmarken
		\item Voraussetzungen:
		\begin{itemize}
			\item Zeichen im Rechtssinn liegt vor
			\newline Zeichen können von Dritten wahrgenommen werden $\rightarrow$ auch Klänge (z.B. Telekom) $\rightarrow$ einzelner Schutz
			\item Wortmarken, Bildmarken, Werbeslogans (kurz kann Marke sein, lang nicht)
			\item Abstrakt geeignete Waren/ Dienstleistungen eines Unternehmens von Waren/ Dienstleistungen anderer Unternehmen zu unterscheiden (nicht von Produkt abhängig): Eignung zur Unterscheidungsfähigkeit
			\item Nicht geeignet Herkunft zu unterscheiden
			\item Kein Zeichen kann von vorneherein ausgeschlossen werden, sehr weit gefasst, erfüllt von fast allem
			\item Abstrakt von Waren abstrahierbar, nicht auf Waren bezogen
		\end{itemize}
	\end{itemize}
	\item Kein Schutzhindernis vorhanden: gemäß §8 (siehe \ref{marke-schutzhindernis})
	\begin{itemize}
		\item Prüfung auf Zusammenhang Marke, Produkt
		\item Konkrete Unterscheidungskraft: für Waren/ Dienstleistungen fehlt jegliche Unterscheidungskraft (geringe Unterscheidungskraft reicht), wenn:
		\begin{itemize}
			\item Verwendungszweck von Produkt wird bezeichnet
			\item Nicht gebräuchliches Wort der deutschen Sprache $\rightarrow$ Unterscheidungskraft
			\item Beschreibenden Angaben fehlt Unterscheidungskraft
			\item Beispiel: "Waschmaschine" für Waschmaschinen
			\item Bezug zum Produkt ("Waschmaschine" für Jeans möglich)
			\item "Auf ersten Blick" unterscheidungskräftig
			\item Wörter, die als Eigenschaftswörter verwendet werden $\rightarrow$ nicht unterscheidungskräftig (nicht "schnell" für Autos)
			\item "Turbo" für Schädlingsbekämpfungsmittel $\rightarrow$ nicht unterscheidungskräftig (synonyme Bedeutung von turbo für schnell, leistungsfähig $\rightarrow$ nicht abstrakt)
			\item "Bonus": Mehrdeutigkeit (Überschuss, Gewinn, Sonderzugabe, Zugabe) macht Unterscheidungsfähigkeit $\rightarrow$ eintragungsfähig
			\item \textcolor{blue}{Übereinstimmung mit Richter nicht wichtig, sondern Nachvollziehbarkeit der Argumentation}
			\item \textcolor{green}{Markieren}: §8.2 gilt nur für eingetragene Marken, nicht für Bentzungsmarken
			\item Fremdsprachige Ausdrücke werden wie deutsche behandelt
			\item Beispiele:
			\begin{itemize}
				\item "individuell" als Kleidungsmarke $\rightarrow$ unterscheidungskräftig
				\item "Look" als Tabakmarke $\rightarrow$ nicht unterscheidungskräftig
			\end{itemize}
			\item Freiheitsbedürfnis:
			\begin{itemize}
				\item Begriffe für Kommunikation gebräuchlich $\rightarrow$ werden nicht geschützt
				\item Merkmale der Waren/ Dienstleistungen bezeichnen $\rightarrow$ nicht geschützt
				\item Begriffe monopolisiert, Andere können Begriff nicht mehr benutzen, um eigene Waren/ Dienstleistungen zu beschreiben (Wettbewerbsfairness)
				\item Blick auf Mitbewerber, Angaben als beschreibender Begriff verwenden
				\item "4You" keinen Warenbezug $\rightarrow$ geschützt
				\item Je größer Freiheitsbedürfnis, desto höher muss Unterscheidungskraft sein
			\end{itemize}
		\end{itemize}
	\end{itemize}
\end{itemize}

\subsection{Entstehung des Markenschutzes} §4
28.04.2022
\begin{itemize}
	\item Eintragung in Register (Markenregister)
	\begin{itemize}
		\item Vorteile Eintragung:
		\begin{itemize}
			\item Nationale Marke, EU-Marke
			\item IR-Marke (internationale Marke): muss 5 Jahre von nationaler Basismarke (eingetragene Marke) abhängig sein
		\end{itemize}
	\end{itemize}
	\item Benutzungs einer Marke im geschäftlichen Verkehr (unabhängig von Eintragung)
	\newline $\rightarrow$ Benutzungsmarke
	\begin{itemize}
		\item Entstehung:
		\begin{itemize}
			\item Nichteintragung einer Marke: meist Nachlässigkeit von kleinen Unternehmen
			\item Eintragung fehlgeschlagen $\rightarrow$ weitere Benutzung
			\newline \textcolor{blue}{Fehlgeschlagene Eintragung bedeutet keine Untersagung der Verwendung, nur direkter Schutz, Schutz kann durch Benutzung als Benutzungsmarke dennoch erreicht werden.}
		\end{itemize}
		\item Innerhalb von Verkehrskreisen Verkehrsgeltung erreicht (Bekanntheit einer Marke)
		\item Hinreichende Bekanntheit führt zu Schutz $\rightarrow$ wie bei Eintragung (gleiche Rechte)
		\item Bekanntheit vor Gericht entschieden
		\item Grundsatz der Priorität
		\item Umfangreiche Benutzung einer Marke (Umfang und Dauer der Benutzung nicht definiert)
		\item empirische Ermittlung (15\%, 37\% Bekanntheit) durch Umfragen
		\begin{itemize}
			\item Kennen Zeichen und Verbinden Marke/ Herkunft damit
			\item Befragung über Zeitpunkt der Anmeldung der eingetragenen Marke
		\end{itemize}
		\item Sehr orginelle Marke benötigt weniger Bekanntheit als weniger orginelle Marke (beschreibt Produkt) um geschützt zu werden
		\item Benutzungsmarken nur regionaler Schutz (Geltungsbereich) $\rightarrow$ nur lokal geschützt
		\newline definiert durch Bekanntheitskreis der Marke $\rightarrow$ kann auch bundesweit geschützt sein
		\item Prozess einer Benutzungsmarke:
		\newline Wenn man schützen möchte, muss man zum Gericht gehen (Umfrage durchführen)
		\item Verkehrsgeltung kann sich ändern (muss ggf. vor Gericht mehrfach verhandelt werden) $\rightarrow$ ohne Verkehrsgeltung kein Schutz mehr
		\item 5 Jahre Zeit gegen eingetragene Marke vorzugehen
		\item Vergleich mit eingetragener Marke:
		\begin{itemize}
			\item Eintragung: immer bundesweiter Schutz:
			\newline Benutzungsschonfrist: muss anfangs nicht zwingend benutzt werden
			\item Benutzungsmarke: kann auch nur regionalen Schutz haben (je nach Bekanntheitsgrad/ -gebiet)
			\item Priorität:
			\begin{itemize}
				\item eingetragene Marke hat häufig höhere Priorität (mit Anmeldungsdatum)
				\newline $\rightarrow$ nicht immer der Fall, da Benutzungsmarke auch länger in Benutzung sein kann
				\newline Gütesiegel der eingetragenen Marke
				\item Benutzungsmarke: Verkehrsgeltung erreichen (dauert länger als der Prozess beim DPMA)
			\end{itemize}
		\end{itemize}
	\end{itemize}
	\item Notorische Bekanntheit (Pariser Verbandsübereinkunft: PVÜ)
	\begin{itemize}
		\item Ausländische Markeninhaber werden wie inländische behandelt
		\item Nahezu jedermann innerhalb der Verkehrskreise bekannt (höhere Ansprüche, 80\% Bekanntheit)
		\item Im Ausland benutzte Marken, die im Inland nicht die entsprechende Bekanntheit für Benutzungsmarke haben
	\end{itemize}
	\item Eintragung und Benutzung haben gleiche Rechte / gleichen Wert
	\item Zeitrang wichtig für Priorität: Benutzungsmarke macht es schwieriger zu Beweisen (wann, welche Bekanntheit)
\end{itemize}

\subsection{Materielle Voraussetzungen der Eintragung einer Marke}
28.04.2022
\begin{itemize}
	\item Materielles Recht §3 und §8
	\item vs. Formelle Voraussetzungen (Formelles Recht)
	\begin{itemize}
		\item z.B. Anträge einreichen
		\item Was muss man tun, um Recht zu bekommen, wenn materielle Voraussetzungen erfüllt sind?
	\end{itemize}
	\item Alle Normen, die Inhalte und Entstehung des Rechts regeln (z.B. BGB)
	\item Von DPMA geprüft
	\item Prüfung von amtswegen
	\item \textcolor{red}{Was sind materielle Voraussetzungen für die Eintragung einer Marke??}
\end{itemize}

\subsection{Die einzelnen Markentypen des § 3 I MarkenG}
28.04.2022
\begin{itemize}
	\item Eingetragene Marke
	\item Benutzungsmarke
\end{itemize}

\subsection{Vom Schutz ausgeschlossene Formen, § 3 II MarkenG} \label{marke-schutzhindernis}
05.05.2022
\begin{itemize}
	\item Absolute vs. relative Schutzhindernisse
	\item §8 Absolute Schutzhindernisse
	\begin{itemize}
		\item gilt nur für Eintragung (hat nichts mit Benutzungsmarken zu tun)
		\item §8.2
		\newline Unterscheidungskraft (§8.2 Abs.1), Freiheitsbedürfnis (\textcolor{red}{§8.2 Abs.2, §8.2 Abs.3?}) (Häufigstes Schutzhindernis)
		\newline Wechselwirkung: beide Absätze liegen zugleich vor
		\item §8.2 Abs.4 Irreführung
		\begin{itemize}
			\item "Marken, [...] die geeignet sind [...] zu täuschen"
			\item z.B. "seit 1958" $\rightarrow$ kann nur falsch sein für neue Marke
			\item z.B. "Pilsener" für Bier, das nicht aus Pilsen kommt
			\item z.B. Dr.-Titel in Markenname
			\newline bei Kosmetika erlaubt $\rightarrow$ Verkehr geht davon aus, dass irgendwer (nicht zwingend Markennamengeber) einen Dr.-Titel hat $\rightarrow$ Wissenschaftliche Grundlage
			\newline Bei anderen Marken oft nicht erlaubt
			\item Fremde Namen kein Schutzhindernis (relatives Schutzhindernis: keine Prüfung auf bereits eingetragen)
			\item Konkret für Fall betrachten, ob Marke in die Irre führt
			\newline $\rightarrow$ hohe Einzelfallgerechtigkeit (theoretisch)
			\newline $\rightarrow$ oft aber Richterabhängig
		\end{itemize}
		\item §8.2 Abs.5 Sittenwidrigkeit
		\begin{itemize}
			\item kommt selten vor
			\item Lebensmittelrechtliche Begriffe
			\item z.B. DDR-Staatswappen: Verstoß gegen Sitten
			\item Verstoß gegen gute Sitten
			\item Individuell zu beurteilen
			\item Religiöse Begriffe (für alltägliche Gegenstände ggf. anstößig)
			\item z.B. "Schlüpferstürmer" $\rightarrow$ Urteil: Verstoß gegen gute Sitten, Frauen werden willenlos gemacht
			\item Verletzung des Schamgefühls, Verletzung gegen religiöse Sitten
			\item Sitten ändern sich auch (zurückhaltend angewandt)
		\end{itemize}
		\item §8.2 Abs.7 offizielle Zeichen
		\begin{itemize}
			\item Historische Kennzeichen sind eintragbar
			\item Muss aktuelles Zeichen sein
		\end{itemize}
		\item §8.3 Formelles Markenrecht
		\begin{itemize}
			\item §8.2 Abs.1, §8.2 Abs.2, §8.2 Abs.3 finden keine Anwendung, wenn Benutzungsmarke
			\begin{itemize}
				\item fehlende Unterscheidungskraft, Freiheitsbedürfnis
				\item Marke kann sich zu Herkunftszeichen durchsetzen und damit eintragungsfähig werden
			\end{itemize}
			\item Verkehrsdurchsetzung vs. Verkehrsgeltung (§4 Entstehung, materielle Schutzvoraussetzungen)
			\newline \textcolor{blue}{Klausurrelevant: Unterschied kennen und beschreiben können}
			\begin{itemize}
				\item Verkehrsgeltung $\rightarrow$ Benutzungsmarke (§4)
				\item Verkehrsdurchsetzung $\rightarrow$ eingetragene Marke
				\newline $\star$ deutlich höherer Bekanntheitsgrad als bei Verkehrsgeltung notwendig
				\newline $\star$ Schutzhindernisse müssen überwunden werden
			\end{itemize}
		\end{itemize}
	\end{itemize}
	\item §3 (absolute Schutzhindernisse)
	\begin{itemize}
		\item §3.1 Abstrakte Markenfähigkeit
		\begin{itemize}
			\item Unterscheidungskraft
			\item vs. konkrete Unterscheidungsfähigkeit (§8.2 Abs.1)
			\newline $\rightarrow$ bezieht sich auf Waren/Dienstleistungen, für die die Marke angemeldet wurde
			\item für irgendein Produkt als unterscheidungsfähig (meist erfüllt)
			\item unabhängig von Produkt
			\item Nicht abstrakt Markenfähig: für kein Produkt Herkunftsfunktion erfüllbar
		\end{itemize}
		\item §3.2 Konkrete Markenfähigkeit
		\begin{itemize}
			\item Produktbezug: technische Wirkung und Art der Ware produktbezogen
			\item Würde besser in §8 passen $\rightarrow$ betrifft Markenfähigkeit und nicht Eintragungsfähigkeit, daher in §3
			\item Standard einer Verpackungsform nicht schutzfähig, orginelle Verpackung kann schutzfähig sein (z.B. Rundes vs. 6-eckiges Glas für Honig schutzfähig)
		\end{itemize}
		\item Lässt mehr Begriffe zu (beispielhafte Aufzählung "insbesondere")
		\item Kein Numerus Klausus der Markenfähigkeit, kein Begriff von vorneherein ausgeschlossen
		\item Wortmarken, Bildmarken: bekannteste
		\begin{itemize}
			\item Buchstaben schutzfähig (früher nicht schutzfähig)
			\begin{itemize}
				\item Buchstaben für Sorten bezeichnen $\rightarrow$ nicht schutzfäig
				\item "C" nicht schutzfähig
				\item "M" für Sportwagen schützenswert
				\newline Keine konkrete Bedeutung für Sportwagen, kein Freiheitsbedürfnis
				\newline "Klasse M" nicht schutzfähig
				\item "T" für Telekommunikation nicht schutzfähig
				\item Mit besonderer Gestaltung schutzfähig als Mischform
			\end{itemize}
			\item Klang / Hörzeichen
			\begin{itemize}
				\item Längere Melodie $\rightarrow$ keine Markenfähigkeit
				\item Wasserrauschen / Hundebellen schutzfähig (nicht für Hundefutter)
				\item Türklingel schutzfähig (nicht für Türen, aber für Hundefutter bspw.)
			\end{itemize}
			\item Formmarken (z.B.: Rolls-Royce Kühlerfigur, Quadratische Form von Schokolade (Rittersport))
			\begin{itemize}
				\item solange kein Ausschlussgrund anch §3.2 vorliegt
				\item Arbeitsbedingtheit: z.B. 3D-Gestaltung nach Grundform der Ware
				\newline z.B. keine naturgetreue Abbildung einer Birne für Birnen
				\newline z.B. keine Kugel für Pralinen (Kugel ist Grundform), wenn unregelmäßig geraspelte Oberfläche (z.B. durch geraspelte Nüsse) $\rightarrow$ schutzfähig
				\newline z.B. Gabelstapler: keine Gabel als Form eintragbar
				\newline z.B. Grundformen nicht schutzfähig
				\item Technisch bedingte Form nicht schutzfähig
				\newline z.B. Anordnung Klingen in Rasierer, technisch bedingt: nicht schutzfähig
				\item Ware nur ästhetische Form, keine Herkunftsfunktion
				\newline z.B. Schmuck
			\end{itemize}
			\item Verpackungsformmarken (z.B.: Coca-Cola Flasche)
			\item Konturlose Farbmarke
			\begin{itemize}
				\item Keine einfache Eintragung (Freiheitsbedürfnis)
				\item Bloße Farbe kaum markenrechtliche Unterscheidungsfähigkeit
				\item Meist Benutzungsmarke $\rightarrow$ Verkehrsdurchsetzung
				\newline \textcolor{red}{(Sollte das nicht eher Verkehrsgeltung sein?)}
				\item Auch Farben können beschreibend sein
				\item \textcolor{red}{Blaupunkt als Marke?}
			\end{itemize}
			\item Geruchsmarke
			\begin{itemize}
				\item Hier weniger eintragungsfähig
				\item Chemische Rezeptur, Geruchsproben eingereicht
				\item Deutsche Gerichte: Nicht ausreichend, in anderen Ländern: eintragungsfähig
				\item Chemische Strukturformel für Laien nicht verständlich $\rightarrow$ daher nicht eintragungsfähig
				\item In EU und Deutschland gibt es keine olfaktorischen marken mehr
				\item Bester Schutz für Riechmarken ist geheim gehaltene Rezeptur
			\end{itemize}
			\item Positionsmarke (z.B.: Knopf im Ohr (Steif))
			\item Bewegungsmarke (z.B.: Brüllender Löwe im Film)
			\item Kennfarbenmarke in Textilindustrie
		\end{itemize}
		\item Bildmarken: schutzfähig
		\begin{itemize}
			\item Beschreibend oder nicht?
			\item Gleiches gilt wie bei Wörtern nur in Bildkommunikation
			\item Stilisierte Bilder (keine naturgetreue Wiedergabe des Produkts)
			\newline Verkehr an stilisierte Bilder gewöhnt
			\item Logos (z.B. Lufthansa-Kranich)
		\end{itemize}
		\item Wortmarken
		\begin{itemize}
			\item Längerer Text $\rightarrow$ keine Markenfähigkeit
			\item Absolutes Schutzhindernis: hauptsächlich beschreibender Begriff
		\end{itemize}
		\item Sachangaben nicht schutzfähig
		\begin{itemize}
			\item z.B. Marke "Kicktennis": Spiele, Turn und Sportartikel $\rightarrow$ nicht schutzfähig (Sachangabe)
			\item z.B. "Stadtwelt": Werbung, Kulturelle Erlebnisse $\rightarrow$ nicht schutzfähig (Sachangabe)
			\item z.B. "Starform": Haarwasser, Dauerwellenflüssigkeit $\rightarrow$ nicht schutzfähig (Sachangabe)
			\item z.B. "Ageless": Mittel zur Körperpflege $\rightarrow$ schutzfähig (keine Sachangabe (stimmt nicht))
			\item z.B. "hearsave": Hörgeräte $\rightarrow$ schutzfähig (keine Sachangabe)
		\end{itemize}
	\end{itemize}
	\item §\textcolor{red}{9??} Relative Schutzhindernisse
	\newline Produktbezug beachten
\end{itemize}

\section{Anmeldestrategien}
\section{Das deutsche Anmeldeverfahren}
\subsection{Anmeldung}
\subsection{Das Verfahren}
\subsection{Rechtsmittel gegen Zurückweisungsbeschlüsse}
\subsection{Umfang des Schutzes}
\subsection{Schranken des Schutzes}
\subsection{Beendigung des Markenschutzes}
\section{Marken im Rechtsverkehr - Nutzung von Markenrechten}
\subsection{Übertragung}
\subsection{Lizensierung}
\section{Schutzmechanismen im Markenrecht}
\subsection{Allgemeines}
\subsection{Voraussetzungen der Schutzansprüche}
\subsection{Häufige Einwendungen des Markeninhabers (= des Widerspruchsgegners)}
\subsection{Markenrechtliche Schutzansprüche von Dritten in Verfahren vor dem DPMA}
\subsection{Verfahren vor den ordentlichen Gerichten}
\section{Sonstige Kennzeichen}
\subsection{Geschäftliche Bezeichnungen}
\subsection{Geografische Herkunftsangaben}
\section{Anmeldungen beim Markenamt in Alicante}
\subsection{Historische Entwicklung des Gemeinschaftsmarkenrechts}
\subsection{Gesetzliche Grundlagen}
\subsection{Grundprinzipien}
\subsection{Eintragungsprinzip und Priorität}
\subsection{Materielles Gemeinschaftsmarkenrecht}
\subsection{Schutzumfang und Schutzdauer}
\subsection{Schranken des Schutzes}
\subsection{Übertragung und Lizenzierung}
\section{Hinterlegung einer Internationalen Registrierung}
\subsection{Das System der internationalen Registrierung}
\subsection{Die Anmeldung}
\subsection{Vergleich Gemeinschaftsmarke / Internationale Registrierung}

\end{document}
